\chapter{Podsumowanie i wnioski}\label{chap:podsumowanie_i_wnioski}
Celem niniejszej pracy było przeprowadzenie analizy porówawczej zróżnicowanej grupy dostępnych neuronowych wizyjnych algorytmów percepcji głębi. W szczególności skupiono się na obiektywnej ocenie ich dokładności, wydajności czasowej, ogólnej efektywności w różnych zastosowaniach oraz wymagań systemowych.

W ramach analizy porównano sześć algorytmów wybranych pod kątem wyników osiąganych w publicznie dostępnych zestawieniach, architektury oraz metody ich uczenia. W celu wykonania rzetelnej analizy wykorzystany został szereg metryk, takich jak pierwiastek ze średniego błądu kwadratowego czy średni bezwzględny błąd procentowy.

W bieżącym rozdziale zawarte jest podsumowanie dokonanej analizy porównawczej oraz wnioski z niej płynące. Przedstawione zostaną kluczowe spostrzeżenia, wskazujące na mocne i słabe strony poszczególnych algorytmów, oraz rekomendacje dotyczące ich zastosowań w praktyce.

\section{Uśrednione wyniki}
W celu usystematyzowania i podsumowania wyników uzyskanych przez algorytmy zostało przygotowane zestawienie średnich wartości każdej z metryk algorytmów uzyskanych na wszystkich zestawach danych. Takie zestawienie pozwala na przejrzyste porównanie efektywności algorytmów i wyciągnięcie jednoznacznych wniosków dotyczących ich wydajności w różnych scenariuszach.

\begin{itemize}
    \item $\delta < 1.25$
    \begin{enumerate}
        \item \textbf{Metric3D}: 0.830
        \item \textbf{DepthAnythingV2}: 0.611
        \item \textbf{MetaPrompt-SD}: 0.436
        \item \textbf{SQLdepth}: 0.300
        \item \textbf{AdelaiDepth}: 0.281
        \item \textbf{GCNDepth}: 0.276
    \end{enumerate}

    \item $\delta < 1.25^2$
    \begin{enumerate}
        \item \textbf{Metric3D}: 0.921
        \item \textbf{DepthAnythingV2}: 0.846
        \item \textbf{MetaPrompt-SD}: 0.636
        \item \textbf{SQLdepth}: 0.300
        \item \textbf{AdelaiDepth}: 0.462
        \item \textbf{GCNDepth}: 0.377
    \end{enumerate}

    \item $\delta < 1.25^3$
    \begin{enumerate}
        \item \textbf{Metric3D}: 0.936
        \item \textbf{DepthAnythingV2}: 0.934
        \item \textbf{MetaPrompt-SD}: 0.762
        \item \textbf{SQLdepth}: 0.532
        \item \textbf{AdelaiDepth}: 0.581
        \item \textbf{GCNDepth}: 0.498
    \end{enumerate}

    \item AbsRel (\%)
    \begin{enumerate}
        \item \textbf{Metric3D}: 11.99
        \item \textbf{DepthAnythingV2}: 28.96
        \item \textbf{MetaPrompt-SD}: 64.67
        \item \textbf{SQLdepth}: 123.82
        \item \textbf{GCNDepth}: 136.50
        \item \textbf{AdelaiDepth}: 244.46
    \end{enumerate}

    \item RMSE
    \begin{enumerate}
        \item \textbf{DepthAnythingV2}: 3.667
        \item \textbf{MetaPrompt-SD}: 4.163
        \item \textbf{Metric3D}: 5.250
        \item \textbf{SQLdepth}: 6.424
        \item \textbf{GCNDepth}: 7.223
        \item \textbf{AdelaiDepth}: 48.031
    \end{enumerate}

    \item czas wykonania predykcji (s)
    \begin{enumerate}
        \item \textbf{GCNDepth}: 0.332
        \item \textbf{AdelaiDepth}: 0.643
        \item \textbf{MetaPrompt-SD}: 0.843
        \item \textbf{SQLdepth}: 0.801
        \item \textbf{DepthAnythingV2}: 1.321
        \item \textbf{Metric3D}: 3.834
    \end{enumerate}
\end{itemize}

\section{Interpretacja wyników}
\subsection{Dokładność estymacji}
W aspekcie dokładności estymacji zdecydowanym liderem jest algorytm Metric3D osiągający najlepszy wynik w większości metrykach dokładności predykcji. Jedynie średnia wartości pierwiastków błędu średniokwadratowego (RMSE) tego algorytmu jest nieoczekiwanie wyższa w porównaniu do pozostałych wiodących algorytmów, jest to jednak spowodowane wysokim wynikiem uzyskanym na syntetycznym zbiorze Taskonomy (21.322) spowodowanym dużą ilością scen zawierających przeszklenia. Algorytm Metric3D bowiem w wielu przypadkach estymuje odległość do przeszklenia podczas gdy zestaw Taskonomy zawiera głębokość do obiektu za przeszkleniem. Model Metric3D jest ponadprzeciętnie dokładny w dużej mierze dzięki implementacji wykorzystania informacji dotyczącej długości ogniskowej kamery. Funkcjonuje poprawnie na scenach zewnętrznych oraz wewnętrznych co świadczy o bardzo wysokiej generalizacji i uniwersalności tego modelu.

Następne miejsce pod kątem dokładności zajął algorytm Depth Anything V2. Należy jednak wziąć pod uwagę fakt, że model ten do funkcjonowania nie wymaga żadnych dodatkowych informacji poza obrazem wejściowym w przeciwieństwie do Metric3D.
