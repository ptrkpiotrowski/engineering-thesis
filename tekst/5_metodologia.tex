\chapter{Metodologia - opis kryteriów oceny algorytmów}\label{chap:metodologia}

W niniejszym rozdziale przedstawiono szczegółowy opis kryteriów oceny neuronowych algorytmów używanych do percepcji głębi. W kontekście niniejszej pracy inżynierskiej istotnym celem jest przeprowadzenie rzetelnej analizy porównawczej, która umożliwi wyłonienie najlepszych rozwiązań w dziedzinie percepcji głębi.

\section{Wybór kryteriów oceny}
Pierwszym etapem było zidentyfikowanie kluczowych kryteriów, które będą decydować o skuteczności i przydatności ocenianych algorytmów. Wybór kryteriów był uzależniony od specyfiki problemu percepcji głębi oraz aktualnych standardów w dziedzinie przetwarzania obrazu. Ostatecznie wybrano następujące kryteria:

\textbf{Dokładność estymacji} - miara precyzyjności algorytmu w określaniu głębi na podstawie danych wejściowych w postaci pojedynczego obrazu RGB. Kryterium dokładności wyrażone zostało z wykorzystaniem najczęściej spotykanych miar dokładności prognozowania, średnim bezwzględnym błądem procentowym \ref{eq:2}, pierwiastkiem ze średniego błądu kwadratowego \ref{eq:3} oraz dokładnością poniżej założonego progu \ref{eq:4}. W poniższych wzorach $ d_p $ oznacza zmierzoną wartość głębi, $ \hat{d}_p $ wartość estymowaną, z kolei $ T $ to ilość pikseli dla których istnieje głębia zmierzona i estymowana.

\begin{equation} \label{eq:2}
    AbsRel = \frac{1}{T} \sum_{p} \frac{|d_p - \hat{d}_p|}{d_p} * 100\%
\end{equation}

\begin{equation} \label{eq:3}
    RMSE = \sqrt{\frac{1}{T} \sum_{p} (d_p - \hat{d}_p)^2}
\end{equation}

\begin{equation} \label{eq:4}
    max(\frac{\hat{d}_p}{d_p}, \frac{d_p}{\hat{d}_p}) = \delta < th
\end{equation}

W celu wyznaczenia wartości miar dokładności wykorzystana została poniższa funkcja, przyjmująca jako argumenty zmierzoną i wyestymowaną przez poszczególne algorytmy mapę głębi. Funkcja ta zwraca na wyjściu słownik z wynikami.

\begin{lstlisting}[style=EEStyle,language=python,numbers=none,frame=none]
def compute_errors(gt, pred):
    mask = get_error_mask(gt)
    gt = gt[mask]
    pred = pred[mask]

    thresh = np.maximum((gt / pred), (pred / gt))
    a1 = (thresh < 1.25).mean()
    a2 = (thresh < 1.25 ** 2).mean()
    a3 = (thresh < 1.25 ** 3).mean()

    abs_rel = np.mean((np.abs(gt - pred) / gt))

    rmse = (gt - pred) ** 2
    rmse = np.sqrt(rmse.mean())

    return dict(a1=a1, a2=a2, a3=a3, abs_rel=abs_rel, rmse=rmse)
\end{lstlisting}

Wykorzystana w niej funkcja \textit{get\_error\_mask} odpowiada za dobranie maski nakładanej na macierze do oceny dokładności. Maski budowane są w kontekście każdego zestawu danych w zestawieniu z konkretną metodą ze względu na możliwości algorytmu oraz maksymalne i minimalne głębokości zarejestrowane w zestawach danych.

\textbf{Szybkość działania} - czas potrzebny przez algorytm na wygenerowanie mapy głębi dla pojedynczego obrazu. Szybkość jest istotnym czynnikiem zwłaszcza w aplikacjach czasu rzeczywistego.

\textbf{Zdolność generalizacji} - zdolność algorytmu do skutecznego przewidywania głębokości scen na nowych, nieznanych wcześniej danych testowych. Jest to istotne kryterium, które mierzy jak dobrze algorytm radzi sobie z różnymi scenariuszami i warunkami, nie tylko tymi na których był uczony.

\textbf{Wymagania sprzętowe} - rzeczywiste zużycie zasobów komputerowych (procesor i pamięć RAM oraz VRAM) podczas predykcji mapy głębi zmierzone narzędziami do monitoringu platformy Google Colab.

\section{Metodologia oceny algorytmów}
Do analizy algorytmów percepcji głębi zastosowano następujące etapy procesu badawczego:
\begin{enumerate}
    \item Implementacja i konfiguracja algorytmów - każdy z algorytmów wytypowanych do analizy został zaimplementowany i skonfigurowany zgodnie z informacjami opisanymi w literaturze oraz repozytoriach poszczególnych metod na platformie Google Colab.
    \item Pomiary i analiza wyników - dla każdego kryterium oceny opisanego w bieżącym rozdziale przeprowadzono szczegółowe pomiary wyników generowanych przez algorytmy. Każdy z algorytmów został przetestowany na zbiorze uczącym oraz przynajmniej jednym niewidzianym w procesie uczenia. Wyniki tych pomiarów zostały zamieszczone na wykresach wspólnych dla wszystkich algorytmów.
    \item Interpretacja i wnioski - na podstawie uzyskanych wyników dokonano interpretacji skuteczności poszczególnych algorytmów w kontekście każdego z kryteriów. Wnioski te stanowią podstawę do rekomendacji oraz możliwych udoskonaleń w przyszłych badaniach.
\end{enumerate}

Założona metodologia oceny algorytmów opiera się na szczegółowym podejściu do definiowania kryteriów oraz ich systematycznej analizie. Przedstawione kryteria stanowią kompleksowy zestaw wskaźników, które umożliwiają obiektywne porównanie i ocenę różnych implementacji technik wizyjnych w kontekście percepcji głebi. Ich zastosowanie pozwala na identyfikację mocnych stron oraz potencjalnych ograniczeń badanych algorytmów, co jest kluczowe dla dalszego rozwoju tej technologii.

\section{Algorytmy podlegające analizie}
Do analizy porównawczej wybranych zostało sześć algorytmów percepcji głebi spośród wszystkich przedstawionych w niniejszej pracy. Selekcja odpowiednich algorytmów polegała na skategoryzowaniu ze względu na rodzaj uczenia danego algorytmu oraz jego architekturę. Z każdej powstałej w ten sposób grupy wybrane zostały algorytmy osiągające najlepsze wyniki w pracach je opisujących oraz publicznie dostępnych zestawieniach. Lista stanowiąca zbiór algorytmów podlegających analizie w dalszej części pracy zamieszczona została poniżej.

\begin{itemize} 
    \item \textbf{AdelaiDepth} - model uczony metodą nadzorowaną oparty o nawracającą sieć neuronową z deklarowanym wynikiem 9\% AbsRel uzyskanym na zbiorze NYU,
    \item \textbf{MetaPrompt-SD} - model uczony metodą nadzorowaną oparty o konwolucyjną sieć neuronową z deklarowanym wynikiem 4,7\% AbsRel uzyskanym na zbiorze KITTI, 
    \item \textbf{Depth Anything V2} - model uczony metodą częściowo nadzorowaną oparty o transformator z deklarowanym wynikiem 4,6\% AbsRel uzyskanym na zbiorze KITTI, 
    \item \textbf{Metric3D} - model uczony metodą nadzorowaną oparty o transformator z deklarowanym wynikiem 3,9\% na ziorze KITTI, 
    \item \textbf{GCNDepth} - model uczony metodą nienadzorowaną oparty o grafową nawracającą sieć neuronową z deklarowanym wynikiem 0,104 AbsRel uzyskanym na zbiorze KITTI,
    \item \textbf{SQLdepth} - model uczony metodą nienadzorowaną oparty o konwolucyjną sieć neuronową z deklarowanym wynikiem 5,2\% AbseRel uzyskanym na zbiorze KITTI.
\end{itemize}