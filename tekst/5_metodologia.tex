\chapter{Metodologia - opis kryteriów oceny algorytmów}\label{chap:metodologia}

W niniejszym rozdziale przedstawiono szczegółowy opis kryteriów oceny neuronowych algorytmów używanych do percepcji głębi. W kontekście niniejszej pracy inżynierskiej istotnym celem jest przeprowadzenie rzetelnej analizy porównawczej, która umożliwi wyłonienie najlepszych rozwiązań w dziedzinie percepcji głębi.

\section{Wybór kryteriów oceny}
Pierwszym etapem było zidentyfikowanie kluczowych kryteriów, które będą decydować o skuteczności i przydatności ocenianych algorytmów. Wybór kryteriów był uzależniony od specyfiki problemu percepcji głębi oraz aktualnych standardów w dziedzinie przetwarzania obrazu. Ostatecznie wybrano następujące kryteria:

\textbf{Dokładność estymacji} - miara precyzyjności algorytmu w określaniu głębokości obiektów na podstawie danych wejściowych w postaci pojedynczego obrazu RGB. Dokładność będzie oceniana na dwóch zbiorach testowych - SYNS-Patches oraz na zbiorze autorskim, żaden obraz z wymienionych zbiorów nie został wykorzystany w uczeniu analizowanych algorytmów. Kryterium dokładności wyrażone zostało z wykorzystaniem najczęściej spotykanych miar dokładności prognozowania, czyli średnim bezwzględnym błądem procentowym \ref{eq:2} oraz średnim błądem kwadratowym \ref{eq:3}. W poniższych wzorach $ d_p $ oznacza zmierzoną wartość głębi, $ \hat{d}_p $ wartość estymowaną, z kolei $ T $ to ilość pikseli dla których istnieje głębia zmierzona i estymowana.

\begin{equation} \label{eq:2}
    AbsRel = \frac{1}{T} \sum_{p} \frac{|d_p - \hat{d}_p|}{d_p}
\end{equation}

\begin{equation} \label{eq:3}
    RMSE = \sqrt{\frac{1}{T} \sum_{p} (d_p - \hat{d}_p)^2}
\end{equation}

W celu wyznaczenia wartości miar dokładności wykorzystana została poniższa funkcja, przyjmująca jako argumenty zmierzoną i wyestymowaną przez poszczególne algorytmy mapę głębi. Funkcja ta zwraca na wyjściu słownik z wynikami.

\begin{lstlisting}[style=EEStyle,language=python,numbers=none,frame=none]
def compute_errors(gt, pred):
    import numpy as np
    mask = gt > 0
    gt = gt[mask]
    pred = pred[mask]
    
    thresh = np.maximum((gt / pred), (pred / gt))
    a1 = (thresh < 1.25).mean()
    a2 = (thresh < 1.25 ** 2).mean()
    a3 = (thresh < 1.25 ** 3).mean()

    abs_rel = np.mean((np.abs(gt - pred) / gt))
    sq_rel = np.mean((((gt - pred) ** 2) / gt))

    rmse = (gt - pred) ** 2
    rmse = np.sqrt(rmse.mean())

    rmse_log = (np.log(gt) - np.log(pred)) ** 2
    rmse_log = np.sqrt(rmse_log.mean())

    err = (np.log(pred) - np.log(gt))
    silog = np.sqrt(np.mean(err ** 2) - np.mean(err) ** 2) * 100

    log_10 = (np.abs(np.log10(gt) - np.log10(pred))).mean()
    return dict(a1=a1, a2=a2, a3=a3, abs_rel=abs_rel, rmse=rmse,
    log_10=log_10, rmse_log=rmse_log, silog=silog, sq_rel=sq_rel)
\end{lstlisting}

\textbf{Szybkość działania} - czas potrzebny przez algorytm na wygenerowanie mapy głębi dla pojedynczego obrazu. Szybkość jest istotnym czynnikiem zwłaszcza w aplikacjach czasu rzeczywistego.

\textbf{Zdolność generalizacji} - zdolność algorytmu do skutecznego przewidywania głębokości scen na nowych, nieznanych wcześniej danych testowych. Jest to istotne kryterium, które mierzy jak dobrze algorytm radzi sobie z różnymi scenariuszami i warunkami, nie tylko tymi na których był uczony.

\textbf{Wymagania sprzętowe} - rzeczywiste zużycie zasobów komputerowych (procesor i pamięć RAM oraz VRAM) podczas predykcji mapy głębi zmierzone narzędziami do monitoringu platformy Google Colab.

\section{Metodologia oceny algorytmów}
Do analizy algorytmów percepcji głębi zastosowano następujące etapy procesu badawczego:
\begin{enumerate}
    \item Implementacja i konfiguracja algorytmów - każdy z algorytmów wytypowanych do analizy został zaimplementowany i skonfigurowany zgodnie z informacjami opisanymi w literaturze oraz repozytoriach poszczególnych metod na platformie Google Colab.
    \item Pomiary i analiza wyników - dla każdego kryterium oceny opisanego w bieżącym rozdziale przeprowadzono szczegółowe pomiary wyników generowanych przez algorytmy. Wyniki te zostały zamieszczone na wykresie wspólnym dla wszystkich algorytmów.
    \item Interpretacja i wnioski - na podstawie uzyskanych wyników dokonano interpretacji skuteczności poszczególnych algorytmów w kontekście każdego z kryteriów. Wnioski te stanowią podstawę do rekomendacji oraz możliwych udoskonaleń w przyszłych badaniach.
\end{enumerate}

Założona metodologia oceny algorytmów opiera się na szczegółowym podejściu do definiowania kryteriów oraz ich systematycznej analizie. Przedstawione kryteria stanowią kompleksowy zestaw wskaźników, które umożliwiają obiektywne porównanie i ocenę różnych implementacji technik wizyjnych w kontekście percepcji głebi. Ich zastosowanie pozwala na identyfikację mocnych stron oraz potencjalnych ograniczeń badanych algorytmów, co jest kluczowe dla dalszego rozwoju tej technologii.
